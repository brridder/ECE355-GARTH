\documentclass{article}
\usepackage{graphicx}
\usepackage{float}
\usepackage{fullpage}
\usepackage{array}
\usepackage{pdflscape}
\usepackage{multirow}
\usepackage{bibentry}
%\nobibliography*
%\usepackage[T1]{fontenc}
%\usepackage{lscape}
%\usepackage[parfill]{parskip}

\setlength{\extrarowheight}{4pt}

\graphicspath{{./images/}}

\floatstyle{boxed}
\restylefloat{figure}

\begin{document}
\input{title.tex}
\tableofcontents
\listoffigures
%\listoftables

\section{Purpose and Scope of Implementation} % Zach
% 10%
% State the purpose of the document and its intended audience. Refer to the
% requirements and design documents.

% Summarize what portion of the design was actually implemented in the
% prototype. Describe the main use cases supported by the implementation. List
% any significant requirements not implemented due to time constraints.

The purpose of this document is to discuss the software implementation that was
created and presented to the evaluators. This software was based on the structure
and guidelines as stated by the software requirements specification document and
software design document that had previously been drafted. This document is
intended to be able to bridge these previous documents with the piece of software
that had been implemented. It will give insight into many of the various design
decisions as well as any changes that were made to the software in which the
implementation had varied from the design as stated in the SDD. It will discuss
testing strategies and results, error handling and what errors were accounted
for, how to install a run the software, as well as the extensibility of the design.

The intended audience of this document are the evaluators of the software project.
They need to be able to make connections with the two previous software documents,
and this document will strive to meet that criteria by discussing the topics mentioned
above. Through reading this implementation report they should be able to see
what subset of the overall design was implemented, what changes were made to
the portions that were indeed implemented, and how to install and run the software
should they feel the need to evaluate it from a closer perspective.

The portion of the design that was implemented was a somewhat small but very
significant portion of the overall proposed design. The implemented design
included a server mock-up with transient storage meant to emulate the server
design. The system controller as presented in the software design document was
realized in the implementation, and communicated with the server and the mobile
applications through the communication stack via sockets. There were two mobile
applications implemented, one of which ran on the Android platform and the other
ran on iOS in order to emulate the cases where a user had either an iPhone or an
Android smartphone. Finally a basic GUI was designed to emulate the triggering
of sensors positioned around the home. These sensors were designed to be able
to replicate the use cases that were implemented. The use cases that were
accounted for originated from the software requirements specification and include:

\begin{itemize}
\item Arming and disarming the system through use of a keypad
\item Door sensor events to detect intruders if system is armed
\item Window intrusion detection if system is armed while a window is opened
\item Irregular thermostat behaviour and over-temperature conditions
\item Flood detection if basement water levels reach an unsuitable level
\item Motion sensors to detect movement in the home while the system is armed
\end{itemize}


\section{Changes to the Design} % Zach
% 10%
% Summarize any changes that were required to the design specified in the
% Design Document, and jusity the design deviations.

\section{Transformation to Implementation} % Ben
% 10%
% Summarize how the design was mapped to the implementation. Consider the
% process model, process communication, storage schema, and other areas were
% implementation decisions were made.

The classes of the system were mapped from the class diagrams in the UML of the
design document in figures of Chapter~5. 

The controllers are run on separate threads so that they can all respond to
events in a timely manner. The event manager, which the controllers and devices
use to communicate through, is run on two separate threads, the main thread and
an additional one, in the same process as the controllers. This was to allow
for the event manager to be able to enqueue events from the controllers without
blocking sending of the events to the controllers.

Database storage was omitted from this release, but there is a simple log of
the events stored with the controllers in JSON format. This log can be easily
parsed into a database with a script once a database has been implemented into
the system.

The process model was kept simple. The controllers talked to the event manager.
The event manager and the devices talked to each other through sockets. A
sensor console user interface was created to assist in sending sensor events
through the sockets to the event manager and eventually the controllers. The
system controller handled what to do for each event type. If an alarm event
needed to be fired off, then the system controller instantiated it and sent it
to the event manager to be sent to the alarm devices and alarm controller.
Every event handled by the system controller was sent to a web server through
JSON-RPC. 

A web server maintained a list of events in memory. These events were fetched
by the two mobile applications for iOS and Android platforms. Events are
viewable on either application depending on what the user has. The iOS
application, due to the overhead required to properly store and display data,
was written in a similar manner to the system itself. Each event type is
represented as a class with proper usage of inheritance. The Android
application, with it's more flexible libraries, was able to use a list data
structure and display the data without first creating models. Both applications
can be easily extended as required by future needs.

\section{Testing Strategy} % Casey 
% 10%
% Identify the general approach taken to testing. For example, discuss the
% manner in which subsystems were tested individually through unit testing and
% in which the system was tested as a whole through integration testing.
% Identify the main subsystems and interfaces included in the test plan.
% Describe the test environment, including emulation of hardware devices.

Several testing strategies were employed during the development of
GARTH. Firstly, test-driven development was used during the initial
phases of development and any time new classes or functionality was
added. Secondly, larger tests were written that verified
interoperability of the various GARTH components. Finally, a sensor
test console program was created for integration testing purposes.



\section{Error Handling} 
% 10%
% Describe potential and realistic faults that could occur in the
% implementation and how the system prevents, detects, mitigates, and/or
% recovers from failure. For instance, describe whether the implementation
% performs error checking on user input, recovers from interrupted
% commmunication, or handles component failure.

\section{Test Results}
% 20%
% Summarize what testing techniques were performed on the implementation.
% Consider tests that pertain to functionality, interfaces, boundary
% conditions, resource handling, performance, load handling, and system
% integration. Do not include full test case descriptions, but rather, describe
% the tests that were executed in general terms.

\section{Outstanding Issues} % Zach
% 10%
% List any outstanding faults, deficiences, or missing features, and make
% recommendations on how they should be addressed in the final version.

\section{Build and Installation Instructions}
% 10%
% Provide a brief outline of the steps required for compiling, installing, and
% running the implementation given a copy of the submitted source code. List
% any development tools and libraries that are required.

\section{Extensibility}
% 10%
% Recommend any functional and non-functional improvments to make in the final
% version of the implementation to be released or a subsequent version. These
% may address use cases that have already been described, or have been
% considered following completion of the Design Document. For instance,
% consider whether the system may be made more reliable, perform better, scale
% to more users, function with new types of components, or be more accessible.
% Give specific examples consistant with the present implementation.

\end{document}
