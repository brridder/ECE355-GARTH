\documentclass{article}
\usepackage{graphicx}
\usepackage{float}
\usepackage{fullpage}
\usepackage{array}
\usepackage{pdflscape}
\usepackage{multirow}
\usepackage{bibentry}
%\nobibliography*
%\usepackage[T1]{fontenc}
%\usepackage{lscape}
%\usepackage[parfill]{parskip}

\setlength{\extrarowheight}{4pt}

\graphicspath{{./images/}}

\floatstyle{boxed}
\restylefloat{figure}

\begin{document}
\begin{titlepage}
\begin{center}
\vfill
\hfill
\\[2cm]
\textsc{\LARGE University Of Waterloo}
\\[1cm]
\textsc{\LARGE ECE 355}
\\[2cm]

\hrule
\hfill
\\[0.5cm]
\textsc{\huge Software Requirements Specification}
\\[0.5cm]
\textsc{\huge GARTH}
\\[0.5cm]
\textsc{\huge Green, Aware, and Responsive Total Home}
\\[0.5cm]
\hrule
\hfill
\\[1cm]
\textsc{\LARGE Group 16} \\[0.4cm]

\begin{minipage}{0.4\textwidth}
\begin{flushleft} \large
Ben Ridder \\
Casey Banner \\
Zack MacLennan
\end{flushleft}
\end{minipage}
\begin{minipage}{0.4\textwidth}
\begin{flushright} \large
brridder \\
20299452 \\
20305946 
\end{flushright}
\end{minipage}


\vfill

{\large \today}
\end{center}
\end{titlepage}

\tableofcontents
\listoffigures
%\listoftables

\section{Purpose and Scope of Implementation} % Zach
% 10%
% State the purpose of the document and its intended audience. Refer to the
% requirements and design documents.

% Summarize what portion of the design was actually implemented in the
% prototype. Describe the main use cases supported by the implementation. List
% any significant requirements not implemented due to time constraints.

\section{Changes to the Design} % Zach
% 10%
% Summarize any changes that were required to the design specified in the
% Design Document, and jusity the design deviations.

\section{Transformation to Implementation} % Ben
% 10%
% Summarize how the design was mapped to the implementation. Consider the
% process model, process communication, storage schema, and other areas were
% implementation decisions were made.

\section{Testing Strategy} % Casey 
% 10%
% Identify the general approach taken to testing. For example, discuss the
% manner in which subsystems were tested individually through unit testing and
% in which the system was tested as a whole through integration testing.
% Identify the main subsystems and interfaces included in the test plan.
% Describe the test environment, including emulation of hardware devices.

\section{Error Handling} 
% 10%
% Describe potential and realistic faults that could occur in the
% implementation and how the system prevents, detects, mitigates, and/or
% recovers from failure. For instance, describe whether the implementation
% performs error checking on user input, recovers from interrupted
% commmunication, or handles component failure.

\section{Test Results}
% 20%
% Summarize what testing techniques were performed on the implementation.
% Consider tests that pertain to functionality, interfaces, boundary
% conditions, resource handling, performance, load handling, and system
% integration. Do not include full test case descriptions, but rather, describe
% the tests that were executed in general terms.

\section{Outstanding Issues} % Zach
% 10%
% List any outstanding faults, deficiences, or missing features, and make
% recommendations on how they should be addressed in the final version.

\section{Build and Installation Instructions}
% 10%
% Provide a brief outline of the steps required for compiling, installing, and
% running the implementation given a copy of the submitted source code. List
% any development tools and libraries that are required.

\section{Extensibility}
% 10%
% Recommend any functional and non-functional improvments to make in the final
% version of the implementation to be released or a subsequent version. These
% may address use cases that have already been described, or have been
% considered following completion of the Design Document. For instance,
% consider whether the system may be made more reliable, perform better, scale
% to more users, function with new types of components, or be more accessible.
% Give specific examples consistant with the present implementation.

\end{document}
