\documentclass{report}

\begin{document}

\tableofcontents

\chapter{Introduction}

Introduction

\chapter{Functional Requirements}

\section{High-level functionality}
The high level functionality of this system can be broken down into several
major sections. Sensors play a major role in interfacing the real world with
the several other sections of the controller. Physical security, both inside
and outside of the home is important. This goes hand in hand with family
security once family members have left the premises of the household. Finally,
green energy management can be easily integrated with the other systems to
ensure prudent usage of resources. 

\subsection{Sensors}
% TODO :: proof read this. It may not all make sense

% TODO :: elaborate HVAC or put it in the glossary
The most crucial part of the system that ties all of the other parts together
is the sensor network. They are the interface between the real world and the
controllers. They detect if the house has been intruded upon by outsiders
through sensors on the windows and doors. They can alert the family if somebody
left the stove top on. Living patterns can be established and monitored to
warn of health problems or to control the HVAC so that energy is not wasted
when nobody is home or they are all asleep. Cameras, motion detectors, hall
effect sensors, and other miscellaneous sensors comprise the majority of this
network.

Cameras play multiple roles in the sensor network. Facial recognition can be
used to detect family members entering and leaving the house to track their
activities. They can be used to alert when strangers are approaching the door
or are on the premises in either a friendly manner or in a more cautious one
depending on the family members currently at home.

Motion detectors provide the security system with the means to trigger alerts 
when motion is detected in certain areas of the home while the system is armed. 
A network of motion detector sensors will be positioned in the hallways of the home, 
with particular attention to entries to the home. Motion detectors outside the 
house will be able to control the lighting system such that people approaching the 
home can be illuminated and easily identified.

Temperature sensors and hall effect sensors are two other major components.
Temperature sensors can be used by both physical security, to ensure that the
oven has not been on for an extended period of time without user interaction as
well as for green energy management to control the overall temperature of the
house or each room. Hall effect sensors can be used with temperature sensors to
detect if windows or doors have been left open when they should be closed in
order to prevent wasting energy. These sensors can also be used to detect
intruders attempting to break in through windows or doors.

\subsection{Family Safety}

In addition to necessary home security functionality, the system will incorporate 
methods to provide safety to members of the users' family. Child safety is a particular 
concern that will be addressed by this portion of the home security system. Sensors 
will be implemented on cabinets as well as certain areas of the home that may contain 
hazardous materials that could be harmful to children. Contact sensors will be present 
on medicine cabinets, closets with cleaning materials, and knife drawers to alert the user
that these areas of the home have been accessed.

\subsection{Security System}

\subsection{Green Energy Management}



\section{Scenarios}



\section{Use case model}

\begin{tabular}{| l | p{7cm} |}
\hline
Use case name & \texttt{NFCDisarmSystem} \\ \hline
Participating Actors & Initiated by \texttt{Resident} \\ \hline
Flow of Events & 

\begin{enumerate}
 \item The \texttt{Resident} enters the home while in possession of an NFC device.
 \item The system begins the disarm countdown.
 \item The \texttt{Resident} approaches the console and holds the NFC within 0.2 meters of the console.
 \item The data on the NFC device is read and validated by the system.
 \item The system enters the disarmed state and the disarm countdown is halted.
\end{enumerate}

\\ \hline

Entry Condition & The system is in the armed state and the Resident
enters the home in possession an NFC device. \\ \hline

Exit Condition & The system is disarmed. \\ \hline
Quality requirements & TODO \\ \hline

\hline
\end{tabular}

\begin{tabular}{| l | p{7cm} |}
\hline
Use case name & \texttt{KeyPadDisarmSystem} \\ \hline
Participating Actors & Initiated by \texttt{Resident} \\ \hline
Flow of Events & 

\begin{enumerate}
 \item The \texttt{Resident} enters the home.
 \item The system begins the disarm countdown.
 \item The \texttt{Resident} approaches the console and enters their code using the keypad on the console.
 \item The entered code is validated by the system.
 \item The system enters the disarmed state and the disarm countdown is halted.
\end{enumerate}

\\ \hline

Entry Condition & The system is in the armed state and the Resident
enters the home. \\ \hline

Exit Condition & The system is disarmed. \\ \hline
Quality requirements & TODO \\ \hline

\hline
\end{tabular}

\begin{tabular}{| l | p{7cm} |}
\hline
Use case name & \texttt{NFCArmSystem} \\ \hline
Participating Actors & Initiated by \texttt{Resident} \\ \hline
Flow of Events & 

\begin{enumerate}
 \item The \texttt{Resident} approaches the console and holds the NFC within 0.2 meters of the console.
 \item The data on the NFC device is read and validated by the system.
 \item The system begins the arm countdown.
 \item The \texttt{Resident} leaves the home.
 \item The system arm countdown completes and the system enters the armed state.
\end{enumerate}

\\ \hline

Entry Condition & The system is in the disarmed state and the \texttt{Resident} wishes to arm it 
using an NFC device. \\ \hline

Exit Condition & The system is armed. \\ \hline
Quality requirements & TODO \\ \hline

\hline
\end{tabular}

\begin{tabular}{| l | p{7cm} |}
\hline
Use case name & \texttt{KeyPadArmSystem} \\ \hline
Participating Actors & Initiated by \texttt{Resident} \\ \hline
Flow of Events & 

\begin{enumerate}
 \item The \texttt{Resident} approaches the console and enters their code.
 \item The entered code is validated by the system.
 \item The system begins the arm countdown.
 \item The \texttt{Resident} leaves the home.
 \item The system arm countdown completes and the system enters the armed state.
\end{enumerate}

\\ \hline

Entry Condition & The system is in the disarmed state and the \texttt{Resident} wishes to arm it
using the keypad. \\ \hline
Exit Condition & The system is armed. \\ \hline
Quality requirements & TODO \\ \hline

\hline
\end{tabular}

\section{Object model}



\section{Dynamic model}



\section{Interfaces}

\chapter{Non-Functional Requirements}

\chapter{User Interfaces}

\end{document}
