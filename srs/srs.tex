\documentclass{report}

\begin{document}

\tableofcontents

\chapter{Introduction}

Introduction

\chapter{Functional Requirements}

\section{High-level functionality}
The high level functionality of this system can be broken down into several
major sections. Sensors play a major role in interfacing the real world with
the several other sections of the controller. Physical security, both inside
and outside of the home is important. This goes hand in hand with family
security once family members have left the premises of the household. Finally,
green energy management can be easily integrated with the other systems to
ensure prudent usage of resources. 

\subsection{Sensors}
% TODO :: proof read this. It may not all make sense

% TODO :: elaborate HVAC or put it in the glossary
The most crucial part of the system that ties all of the other parts together
is the sensor network. They are the interface between the real world and the
controllers. They detect if the house has been intruded upon by outsiders
through sensors on the windows and doors. They can alert the family if somebody
left the stove top on. Living patterns can be established and monitored to
warn of health problems or to control the HVAC so that energy is not wasted
when nobody is home or they are all asleep. Cameras, motion detectors, hall
effect sensors, and other miscellaneous sensors comprise the majority of this
network.

Cameras play multiple roles in the sensor network. Facial recognition can be
used to detect family members entering and leaving the house to track their
activities. They can be used to alert when strangers are approaching the door
or are on the premises in either a friendly manner or in a more cautious one
depending on the family members currently at home.

Motion detectors 

Temperature sensors and hall effect sensors are two other major components.
Temperature sensors can be used by both physical security, to ensure that the
oven has not been on for an extended period of time without user interaction as
well as for green energy management to control the overall temperature of the
house or each room. Hall effect sensors can be used with temperature sensors to
detect if windows or doors have been left open when they should be closed in
order to prevent wasting energy. These sensors can also be used to detect
intruders attempting to break in through windows or doors.

\subsection{Family Safety}

\subsection{Security System}

\subsection{Green Energy Management}



\section{Scenarios}



\section{Use case model}



\section{Object model}



\section{Dynamic model}



\section{Interfaces}

\chapter{Non-Functional Requirements}

\chapter{User Interfaces}

\end{document}
